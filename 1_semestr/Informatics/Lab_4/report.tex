\documentclass[12pt,onecolumn]{article}

\usepackage{listings}
\usepackage{float}
\usepackage{mathtools}
\usepackage[russian]{babel}
\everymath{\displaystyle}

\usepackage[usenames]{color}
\usepackage{colortbl}

\usepackage{geometry}
\usepackage{minted}
\geometry{
  a4paper,
  top=15mm, 
  right=10mm, 
  bottom=15mm, 
  left=10mm
}

\definecolor{dkgreen}{rgb}{0,0.6,0}
\definecolor{gray}{rgb}{0.5,0.5,0.5}
\definecolor{mauve}{rgb}{0.58,0,0.82}

\lstset{frame=tb,
  language=sh,
    aboveskip=3mm,
      belowskip=3mm,
        showstringspaces=false,
	  columns=flexible,
	    basicstyle={\small\ttfamily},
	      numbers=none,
	        numberstyle=\tiny\color{gray},
		  keywordstyle=\color{blue},
		    commentstyle=\color{dkgreen},
		      stringstyle=\color{mauve},
		        breaklines=true,
			  breakatwhitespace=true,
			    tabsize=3
			    }
\begin{document}

\begin{center}
    Федеральное государственное автономное образовательное учреждение высшего образования\\
	«Национальный исследовательский университет ИТМО»
\end{center}
\vspace{1cm}
\setcounter{page}{0} 
\begin{center}
    \large \textbf{Отчет}\\
    \textbf{по лабораторной работе №4}\\
    \large \textbf{«Исследование протоколов, форматов обмена информацией и языков разметки документов»}\\
     по дисциплине «Информатика»\\
	\vspace{1cm}
    Вариант №4\\
\end{center}

\vspace{10cm}
\begin{flushright}
  Выполнил: Левченко Ярослав Алексеевич, группа P3118\\
  Преподаватель: Рыбаков Степан Дмитриевич\\
\end{flushright}

\vspace{5cm}
\begin{center}
    г. Санкт-Петербург\\
    2022г.
\end{center}
\thispagestyle{empty}
\newpage
\tableofcontents
\newpage
\section{Задание 1}
Написать программу на языке Python 3.x, которая бы осуществляла парсинг и конвертацию исходного файла в новый. \\
Решение: \\
https://github.com/aufsell/ITMO\_1\_SEMESTR/tree/main/Informatics/Lab\_4
\newpage
\section{Задание 2}
a) Найти готовые библиотеки, осуществляющие аналогичный парсинг и конвертацию файлов. \\
b) Переписать исходный код, применив найденные библиотеки. Регулярные выражения также нельзя использовать. \\
c) Сравнить полученные результаты и объяснить их сходство/различие. \\
\inputminted{python}{trash/1}
Различия:\\
- Форматирование \\
- Формат представление массива \\
\newpage
\section{Задание 3}
a) Переписать исходный код, добавив в него использование регулярных выражений. \\
b) Сравнить полученные результаты и объяснить их сходство/различие. \\
Различия между данным решением и первым отсутсвуют \\
\inputminted{python}{trash/2}
\newpage 
\section{Задание 4}
a) Используя свою исходную программу из обязательного задания, программу из дополнительного задания No1 и программу из дополнительного задания No2, сравнить стократное время выполнения парсинга + конвертации в цикле. \\
b) Проанализировать полученные результаты и объяснить их сходство/различие. \\
Свое - 0.001935125001182314 \\
Библиотеки - 0.000519692079978995 \\
Регулярные выражения - 0.0029144170039216988 \\
Библиотечная реализация работает быстрее всего из-за более грамотной реализации лексера, а так же нерекурсивного построения дерева xml \\
Решение через регулярные выражения самое медленное из-за многократного построения деревьев регулярного выражения
\newpage 
\section{Задание 5}
c) Переписать исходную программу, чтобы она осуществляла парсинг и конвертацию исходного файла в любой другой формат (кроме JSON, YAML, XML, HTML): PROTOBUF, TSV, CSV, WML и т.п. \\
\inputminted{python}{trash/3}
\newpage 
\section{Вывод}
По ходу выполнения данной лабораторной работы, я узнал о форматах представления данных (JSON,XML,CSV), а так же научился парсить данные форматы, как с помощью готовых решений, так и самостоятельно. 
\newpage 
\section{Список литературы}
- Лямин А.В., Череповская Е.Н. Объектно-ориентированное программирование. Компьютерный практикум. – СПб: Университет ИТМО, 2017. – 143 с. – Режим доступа: https://books.ifmo.ru/file/pdf/2256.pdf.
- Пишем изящный парсер на Питоне - Режим доступа: https://habr.com/en/post/309242/
\end{document}